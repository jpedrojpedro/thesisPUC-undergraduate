\documentclass[graduacao,brazil]{ThesisPUC}

%%%%%%%%%%%%%%%%%%%%%%%%%%%%%%%%%%%%%%%%%%%%%%%%%%%%%%%%%%%%%%%%%%%%%%%%%%%%%%%%

\newcommand{\Rset}{\mathbb{R}}
\newcommand{\Zset}{\mathbb{Z}}

%%%%%%%%%%%%%%%%%%%%%%%%%%%%%%%%%%%%%%%%%%%%%%%%%%%%%%%%%%%%%%%%%%%%%%%%%%%%%%%%

\autor{Jo\~{a}o Pedro Vallad\~{a}o Pinheiro}
\autorR{Jo\~{a}o Pedro, Pinheiro}
\orientador{S\'{e}rgio Lifschitz}
\orientadorR{Lifschitz, S\'{e}rgio}

\titulo{Plataforma de Estudo para Aux\'{i}lio no Aprendizado de Banco de Dados}
\titulouk{Study Platform to Aid Database Learning}
\dia{16} \mes{Janeiro} \ano{2014}

\cidade{Rio de Janeiro}
\CDD{510}
\departamento{Inform\'{a}tica}
\programa{do Curso de Engenharia de Computa\c{c}\~{a}o}
\centro{Centro T\'{e}cnico Cient\'{i}fico}
\universidade{Pontif\'{i}cia Universidade Cat\'{o}lica do Rio de Janeiro}
\uni{PUC--Rio}
\course{Engenharia de Computa\c{c}\~{a}o}
\diploma{t\'{i}tulo de Engenheiro de Computa\c{c}\~{a}o}

%%%%%%%%%%%%%%%%%%%%%%%%%%%%%%%%%%%%%%%%%%%%%%%%%%%%%%%%%%%%%%%%%%%%%%%%%%%%%%%%

%TODO: Agradecer Vanessa, JMilk, JJ e Daniel
\agradecimentos{
Agrade\c{c}o ao apoio incondicional de toda a minha fam\'{i}lia em todos os momentos da minha vida.
}

%%%%%%%%%%%%%%%%%%%%%%%%%%%%%%%%%%%%%%%%%%%%%%%%%%%%%%%%%%%%%%%%%%%%%%%%%%%%%%%%

\chaves{
  \chave{Plataforma de Estudo}
  \chave{Banco de Dados}
  \chave{SQL}
}

\resumo{
Elabora\c{c}\~{a}o de uma plataforma web que auxilie o aprendizado do aluno da disciplina Banco de
Dados. O tema surgiu a partir do conceito flipped classroom, que se trata de um modelo que
sugere o aprendizado online. O aluno assiste, pratica e discute determinados assuntos em casa,
trazendo as d\'{u}vidas para a sala de aula. Dessa forma, as aulas tornam-se mais din\^{a}micas e
menos expositivas.
}

%%%%%%%%%%%%%%%%%%%%%%%%%%%%%%%%%%%%%%%%%%%%%%%%%%%%%%%%%%%%%%%%%%%%%%%%%%%%%%%%

\chavesuk{
  \chave{Study Platform}
  \chave{Database}
  \chave{SQL}
}

\resumouk{
Elaboration of web platform which aids student learning of database discipline. The topic came up
from flipped classroom concept, whose model suggests online learning. The student attends,
practice and discusses certain subjects at home, bringing doubts to classroom. Thus, classes
became more dynamic and less expository.
}

%%%%%%%%%%%%%%%%%%%%%%%%%%%%%%%%%%%%%%%%%%%%%%%%%%%%%%%%%%%%%%%%%%%%%%%%%%%%%%%%

\modotabelas{figtab} % nada, fig, tab ou figtab

%%%%%%%%%%%%%%%%%%%%%%%%%%%%%%%%%%%%%%%%%%%%%%%%%%%%%%%%%%%%%%%%%%%%%%%%%%%%%%%%

\begin{document}

\chapter{Introdu\c{c}\~{a}o}

Atrav\'{e}s do advento e populariza\c{c}\~{a}o da internet, pessoas envolvidas com educa\c{c}\~{a}o
encontraram uma poss\'{i}vel solu\c{c}\~{a}o para d\'{e}ficit de ensino global. A diminui\c{c}\~{a}o
das dist\^{a}ncias entre educadores e alunos gerou uma gama de poss\'{i}veis servi\c{c}os a serem prestados
em prol de um melhor ensino.

A partir disso, algumas iniciativas foram desenvolvidas e est\~{a}o inseridas com sucesso no
mercado. Foram contempladas desde ideias sem fins lucrativos, como \textbf{Edx} e \textbf{Khanacademy}
(ambos serão explicadas em detalhes na seção 2.1), at\'{e} startups que oferecem parte dos cursos
gratuitamente e outra parte paga, como o caso da \textbf{Code School} (tamb\'{e}m explicada em maiores
detalhes na seção 2.1).

Essa metodologia de ensino a dist\^{a}ncia recebeu um nome: \textbf{flipped classroom}. Trata-se de
uma invers\~{a}o nos padr\~{o}es de ensino adotados pelas escolas, no qual a passagem de
conhecimento passa a ser online, atrav\'{e}s de v\'{i}deos de curta dura\c{c}\~{a}o, e a sala de aula torna-se
um ambiente de resolu\c{c}\~{a}o de exerc\'{i}cios.

\begin{figure}[h]
    \includegraphics*[width=\linewidth]{Imagens/flipped_classroom.png}
    \caption{Inversão nos padrões de ensino sugerida pelo modelo flipped classroom}
\end{figure}

\chapter{Proposta}

O presente trabalho tem como objetivo desenvolver uma ferramenta, de cunho acad\^{e}mico
e opensource, que auxilie o ensino de banco de dados. O escopo do projeto abranger\'{a} o ensino
da linguagem SQL, atrav\'{e}s da disponibiliza\c{c}\~{a}o de exerc\'{i}cios online e f\'{o}rum para discuss\~{a}o dos
mesmos ou de assuntos referentes \''{a} disciplina. Al\'{e}m disso, todas as listas de exerc\'{i}cios contidas
no site ser\~{a}o, em um primeiro momento, as mesmas oferecidas na disciplina de Banco de Dados
da PUC--Rio (INF 1383).

Atualmente, os alunos da disciplina podem resolver os exerc\'{i}cios propostos em algumas
aulas pr\'{a}ticas, durante hor\'{a}rio de aula, ou utilizar um programa desktop que se conecta ao
servidor da disciplina com o \textbf{SGBD} j\'{a} configurado. Os exerc\'{i}cios est\~{a}o vinculados, na maioria das
vezes, a um ou mais esquemas contidos no SGBD em quest\~{a}o. Dessa forma, os alunos acessam
um mesmo esquema para tentar solucionar determinada quest\~{a}o. Quando trata-se de uma lista de
exerc\'{i}cios que envolve puramente o comando \textbf{SELECT}, n\~{a}o h\'{a} maiores problemas quanto ao
esquema compartilhado. Por\'{e}m, quando h\'{a} comandos \textbf{DML} do tipo \textbf{INSERT}, \textbf{UPDATE} ou
\textbf{DELETE}, ou comandos \textbf{DDL} envolvidos na resolu\c{c}\~{a}o da lista, pode-se gerar uma
indisponibiliza\c{c}\~{a}o moment\^{a}nea, caso um comando seja executado de forma incorreta.

Sendo assim, duas das principais caracter\'{i}sticas do sistema proposto ser\~{a}o a utiliza\c{c}\~{a}o de
esquemas distintos por alunos a cada quest\~{a}o e a possibilidade de voltar em uma determinada
modifica\c{c}\~{a}o feita anteriormente no esquema. Logo, al\'{e}m de ter liberdade de executar qualquer
comando na base, sabendo que n\~{a}o interferir\'{a} os demais alunos, o aluno poder\'{a} voltar em um
estado anterior da base quando bem entender.

\section{Avalia\c{c}\~{o}es dos Concorrentes -- An\'{a}lise SWOT}

A \textbf{An\'{a}lise SWOT} \'{e} um sistema simples para posicionar ou verificar a posi\c{c}\~{a}o estrat\'{e}gica
da empresa no ambiente em quest\~{a}o. O termo SWOT \'{e} uma sigla oriunda dos termos ingleses
Strenghts (Forças), Weaknesses (Fraquezas), Opportunities (Oportunidades) e Threats (Ameaças).

-- Strengths (forças) - vantagens internas da empresa em relação às concorrentes. Ex.:
qualidade do produto oferecido, bom serviço prestado ao cliente, solidez financeira, etc.

-- Weaknesses (fraquezas) - desvantagens internas da empresa em relação às concorrentes.
Ex.: altos custos de produção, má imagem, instalações desadequadas, marca fraca, etc.;

-- Opportunities (oportunidades) – aspectos externos positivos que podem potenciar a
vantagem competitiva da empresa. Ex.: mudanças nos gostos dos clientes, falência de
empresa concorrente, etc.;

-- Threats (ameaças) - aspectos externos negativos que podem por em risco a vantagem
competitiva da empresa. Ex.: novos competidores, perda de trabalhadores fundamentais,
etc.

\begin{figure}[h]
    \includegraphics*[width=\linewidth]{Imagens/swot.jpg}
    \caption{Ilustração que demonstra todos os critérios utilizados na Análise SWOT}
\end{figure}

\chapter{Projeto}

\chapter{Plataforma Web}

\chapter{Consideracoes Finais}

\chapter{Refer\^{e}ncias Bibliogr\'{a}ficas}

\arial
\bibliography{Exemplo}

\normalfont

\end{document}